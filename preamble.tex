
\documentclass[
	a4paper, % Page size
	fontsize=10pt, % Base font size
	twoside=true, % Use different layouts for even and odd pages (in particular, if twoside=true, the margin column will be always on the outside)
	%open=any, % If twoside=true, uncomment this to force new chapters to start on any page, not only on right (odd) pages
	%chapterentrydots=true, % Uncomment to output dots from the chapter name to the page number in the table of contents
	numbers=noenddot, % Comment to output dots after chapter numbers; the most common values for this option are: enddot, noenddot and auto (see the KOMAScript documentation for an in-depth explanation)
]{kaobook}

\usepackage{mathtools}
\usepackage{cancel}
% Choose the language
\ifxetexorluatex
	\usepackage{polyglossia}
	\setmainlanguage{english}
\else
	\usepackage[english]{babel} % Load characters and hyphenation
\fi
\usepackage[english=british]{csquotes}	% English quotes

% Load packages for testing
\usepackage{blindtext}
%\usepackage{showframe} % Uncomment to show boxes around the text area, margin, header and footer
%\usepackage{showlabels} % Uncomment to output the content of \label commands to the document where they are used

% Load the bibliography package
\usepackage{kaobiblio}
\addbibresource{main.bib} % Bibliography file

% Load mathematical packages for theorems and related environments
\usepackage[framed=true]{kaotheorems}

% Load the package for hyperreferences
\usepackage{kaorefs}

\graphicspath{{examples/documentation/images/}{images/}} % Paths in which to look for images

\makeindex[columns=3, title=Alphabetical Index, intoc] % Make LaTeX produce the files required to compile the index

\makeglossaries % Make LaTeX produce the files required to compile the glossary
\input{glossary.tex} % Include the glossary definitions

\makenomenclature % Make LaTeX produce the files required to compile the nomenclature

% Reset sidenote counter at chapters
%\counterwithin*{sidenote}{chapter}

% --------------------------------------------
% 		SOME MATH COMMANDS DEFINITIONS
% --------------------------------------------


\renewcommand{\d}{\text d} % differential
\newcommand{\dd}[2]{\dfrac{\d #1}{\d #2}} % total derivative 
\newcommand{\ddp}[2]{\dfrac{\partial #1}{\partial #2}} % partial derivative
\renewcommand{\vec}[1]{ \mathbf #1} % vectors with bold font
\newcommand{\abs}[1]{\lvert #1 \rvert} % absolute value
\newcommand{\del}{\partial} % del operator is not nabla
\newcommand{\paren}[1]{\left(#1\right)} % big parentheses
\newcommand{\commu}[2]{#1 #2 - #2 #1} % expansion of commutator
\newcommand{\ket}[1]{\lvert #1 \rangle} % ket
\newcommand{\bra}[1]{\langle #1 \rvert} % bra
\newcommand{\normbraket}[1]{\langle #1 \rvert #1 \rangle} % the norm of something in bra-ket notation
\newcommand{\then}{\Longrightarrow} % then arrow
%\newcommand{\def}{\coloneqq} % definition denoted as :=
\renewcommand{\mapsto}{\longmapsto} % long arrow for maps-to symbol
\newcommand{\reals}{\mathbb R} % set of real numbers
\newcommand{\complexes}{\mathbb C} % set of complex numbers