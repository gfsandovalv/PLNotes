\documentclass[
	a4paper, % Page size
	fontsize=10pt, % Base font size
	twoside=true, % Use different layouts for even and odd pages (in particular, if twoside=true, the margin column will be always on the outside)
	%open=any, % If twoside=true, uncomment this to force new chapters to start on any page, not only on right (odd) pages
	%chapterentrydots=true, % Uncomment to output dots from the chapter name to the page number in the table of contents
	numbers=noenddot, % Comment to output dots after chapter numbers; the most common values for this option are: enddot, noenddot and auto (see the KOMAScript documentation for an in-depth explanation)
]{kaobook}

\usepackage[utf8]{inputenc}
\usepackage[T1]{fontenc}
\usepackage{amsmath}
\usepackage{amsfonts}
\usepackage{amssymb}
\usepackage[version=4]{mhchem}
\usepackage{stmaryrd}
\usepackage{graphicx}
\usepackage[export]{adjustbox}
\usepackage[english]{babel}
\usepackage{mathtools}
\usepackage{cancel}

% Choose the language

%\ifxetexorluatex
%	\usepackage{polyglossia}
%	\setmainlanguage{english}
%\else
%	\usepackage[english]{babel} % Load characters and hyphenation
%\fi
\usepackage[english=british]{csquotes}	% English quotes

% Load packages for testing
\usepackage{blindtext}
%\usepackage{showframe} % Uncomment to show boxes around the text area, margin, header and footer
%\usepackage{showlabels} % Uncomment to output the content of \label commands to the document where they are used

% Load the bibliography package
\usepackage{kaobiblio}
\addbibresource{main.bib} % Bibliography file

% Load mathematical packages for theorems and related environments
\usepackage[framed=true]{kaotheorems}

% Load the package for hyperreferences
\usepackage{kaorefs}

\makeindex[columns=3, title=Alphabetical Index, intoc] % Make LaTeX produce the files required to compile the index

\makeglossaries % Make LaTeX produce the files required to compile the glossary
\newglossaryentry{computer}{
	name=computer,
	description={is a programmable machine that receives input, stores and manipulates data, and provides output in a useful format}
}

% Glossary entries (used in text with e.g. \acrfull{fpsLabel} or \acrshort{fpsLabel})
\newacronym[longplural={Frames per Second}]{fpsLabel}{FPS}{Frame per Second}
\newacronym[longplural={Tables of Contents}]{tocLabel}{TOC}{Table of Contents}

 % Include the glossary definitions

\makenomenclature % Make LaTeX produce the files required to compile the nomenclature

% Reset sidenote counter at chapters
%\counterwithin*{sidenote}{chapter}

% --------------------------------------------
% 		DEFINICIONES DE COMANDOS Y SÍMBOLOS MATEMÁTICOS
% --------------------------------------------

\renewcommand{\d}{\text d} % diferencial, una d sin itálicas 
\newcommand{\dd}[2]{\dfrac{\d #1}{\d #2}} % derivada total de #1 respecto a #2
\newcommand{\ddp}[2]{\dfrac{\partial #1}{\partial #2}} % derivada parcial de #1 respectoa a #2
\renewcommand{\vec}[1]{ \mathbf #1} % vectores con negrita en vez de la flecha como acento 
\newcommand{\abs}[1]{\lvert #1 \rvert} % valor absoluto
\newcommand{\del}{\partial} % del := símbolo de derivada parcial 
\newcommand{\paren}[1]{\left(#1\right)} % parentesis grandes, se adaptan al tamaño de la expresión
\newcommand{\commu}[2]{#1 #2 - #2 #1} % expansión de un conmutador
\newcommand{\bra}[1]{\langle #1 \rvert} % bra
\newcommand{\ket}[1]{\lvert #1 \rangle} % ket
\newcommand{\normbraket}[1]{\langle #1 \rvert #1 \rangle} % la norma de un estado en notación de dirac
\newcommand{\matrixElement}[3]{\bra{#2} #1 \ket{#3}} % elemento de matriz de un operador <#2|#1|#3> 
\newcommand{\then}{\Longrightarrow} % Flecha de "entonces"
%\newcommand{\def}{\coloneqq} % definición de una variable/símbolo con ":=" 
\renewcommand{\mapsto}{\longmapsto} % flecha larga para indicar el mapeo de una funcipon f: A ---> B 
\newcommand{\reals}{\mathbb R} % Números reales
\newcommand{\complexes}{\mathbb C} % Números complejos

% --------------------------
%       PATH TO IMAGES
% --------------------------

\graphicspath{ 
    {chapters/}
    {chapters/matematicas/sistema_de_coordenadas/}
}%

