# theorems about commuting operators


> [!Theorem 1]
> If two operators $A$ and $B$ commute, and if $\ket{\psi}$ is an eigenvector of $A$. Then $B \ket{\psi}$ is also an eigenvector of $A$ with the same eigenvalue.

> [!Proof 1] 
> Since $\ket{\psi}$ is eigenvector of $A$ it satisfies $A\ket{\psi} = a\ket{\psi}$. Apliying $B$ over the state on both sides one gets $BA\ket{\psi} = aB\ket{\psi}$. By hypothesis $AB = BA$, then $A(B\ket{\psi}) = a(B\ket{\psi})$. This last one means that the state $\ket{\phi} = B\ket{\psi}$ is eigenvector of $A$ with eigenvalue $a$.
> In the case $a$ is non-degenerate...
> #TODO continue with the proof

> [!Theorem 2]
> If two operators $A$ and $B$ commute, and if $\ket{ \psi_{1}}$ and $\ket{ \psi_{2}}$ are two eigenvectors of $A$ with different eigenvalues, the matrix element $\braketB{B}{\psi_{1}}{\psi_{2}}$ is zero.

> [!Proof 2]
> Contents


> [!Theorem 3]
> If two observables $A$ and $B$ commute, one can construct an orthonormal basis of the state space with eigenvectors common to $A$ and $B$.

# References

[QM, Cohen - Ch 2](file:///home/gabo/Zotero/storage/7DZ9SFA4/Cohen-Tannoudji%20et%20al.%20-%202020%20-%20Quantum%20mechanics.%20Volume%201%20Basic%20concepts,%20tools.pdf)


 

