\section{Teoremas sobre operadores que conmutan}

\begin{theorem}
    If two operators $A$ and $B$ commute, and if $\ket{\psi}$ is an eigenvector of $A$. Then $B \ket{\psi}$ is also an eigenvector of $A$ with the same eigenvalue.
\end{theorem}

Since $\ket{\psi}$ is eigenvector of $A$ it satisfies $A\ket{\psi} = a\ket{\psi}$. Apliying $B$ over the state on both sides one gets $BA\ket{\psi} = aB\ket{\psi}$. By hypothesis $AB = BA$, then $A(B\ket{\psi}) = a(B\ket{\psi})$. This last one means that the state $\ket{\phi} = B\ket{\psi}$ is eigenvector of $A$ with eigenvalue $a$.
In the case $a$ is non-degenerate...
%TODO continue with the proof

\begin{theorem}
    Si dos operadores $A$ y $B$ conmutan, y si $\ket{ \psi_{1}}$ y $\ket{ \psi_{2}}$ son dos autovectores de $A$ con diferentes autovalores, se tiene que el elemento de matriz $\matrixElement{B}{\psi_{1}}{\psi_{2}}$ es nulo.
\end{theorem}


\begin{theorem}
    Si dos observables representados por los operadores $A$ y $B$ conmutan, se puede construir una base ortonormal de espacio de estados con autovectores comunes a $A$ y $B$.
\end{theorem}
References




 

